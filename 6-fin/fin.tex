\section{結言}
超音波照射された擬塑性流体中の落下球の高速化現象に関して,理論の妥当性を調査した.密度差,濃度,球径をパラメータとして実験した.以下に実験結果と得られた知見を示す.

\begin{itemize}
    \item 溶液と落下球の超音波照射による高速化度合と密度差の関係は,PAA濃度0.5,0.7wt.\%においては正の相関が,PAA濃度1.0wt.\%においては負の相関がみられた.理論では終端速度が遅くなる場合と,密度差が小さい場合に高速化が顕著となるものの,満たした場合はPAA濃度1.0wt.\%のみであった.これより,密度差以外による要因が大きいためと理論が適用できないと示唆される.
    \item 溶液の濃度を増加させた場合,落下球の終端速度は遅くなり,高速化がより顕著となる.しかし,濃度が高くなると高速化が抑制される.PAA溶液0.7-1.0wt.\%が高速化の極大となる.理論では溶液濃度が高い場合,高速化が顕著となると示される.これより,PAA濃度1.0wt.\%より溶液濃度が高い場合,理論が適用できないことが示唆された.
    \item PAA濃度1.0wt\%,落下球が鋼球の場合において直径を変化させた場合,直径10mmが高速化の極大となる.直径が10mmより大きい場合,高速化度合は粘度比と音響境界層厚さを落下球の半径で規格化した値の積と正の相関関係が見られる.一方で,球径が10mmより小さい場合,理論が適用できないことが示唆された.
    \item 本研究の実験条件において,高速化度合は粘度比と音響境界層厚さを球の半径で規格化した値の積が0.2以下において,正の相関がみられた.それ以上の範囲では高速化度合と粘度比と音響境界層厚さを球の半径で規格化した値との積に相関がみられなかった.応力比を用いて考えると,応力比1近傍にて高速化が顕著にみられ,それ以上では抑制されていた.これより,弾性影響が高速化を抑制することが分かった.
\end{itemize}

上記の結果から,超音波照射された擬塑性流体中の落下球の高速化現象に関して,理論が適用できる条件には限りがあることが分かった.弾性影響が無視できない粘弾性流体において,終端速度が遅い場合,理論が適用できないことが示唆された.弾性影響下で高速化現象を示すためには,弾性の影響を考慮した理論の構築が必要であると考えられる.