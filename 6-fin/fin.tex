\section{結言}
本研究において,擬塑性流体中を落下する球へ超音波照射することによって落下速度が高速化する現象の要因に関して,高速化の抑制要因に関して調査をした.これらの調査を行うため,密度差,濃度,球径をパラメータとして実験を行った.以下に実験の結果,得られた知見を示す.

\begin{itemize}
    \item 溶液と落下球の密度差が小さくなると,落下球の終端速度は遅くなり,超音波照射による高速化がより顕著となる.これは,落下球によって生じる周囲流体の粘性による影響であると考えられる.
    \item 溶液の濃度を上昇させた場合,落下球の終端速度は遅くなり,高速化がより顕著となる.これは,溶液の濃度が濃くなることで落下球が,流体より受ける粘性力が強くなり,流体の擬塑性も強くなるためだと考えられる.しかし,濃度が高くなると高速化が抑制される.PAA溶液0.7-1.0wt.\%が高速化のピークとなる.
    \item PAA溶液1wt\%,落下球が鋼球の場合において直径を変化させた場合,直径10[mm]が高速化のピークとなる.直径が10[mm]より大きい場合,球の落下による粘度と音響境界層内粘度の比に落下球の半径で規格化した音響境界層厚さを乗じた値と高速化度合いは正の相関関係となる.これは球径が大きいと落下速度が早くなり,落下による周囲流体の粘性による影響が弱くなるためだと考えられる.
    \item 本研究の実験条件において,球の落下による粘度と音響境界層内粘度の比に落下球の半径で規格化した音響境界層厚さを乗じた値が0.1以下において,高速化度合いと正の相関を取った.一方でそれ以上の領域では一旦高速化が抑制され,粘度比が再度大きくなると高速化が見られるようになったが,その傾きは小さいものであった.応力比を用いて考えると,応力比1近傍にて高速化が顕著にみられ,それ以上では抑制されていた.これより,弾性による影響が支配的となると高速化が抑制されることが分かった.
\end{itemize}

上記の結果から,擬塑性流体中を落下する球へ超音波照射すると落下速度が高速化する現象に関して,落下球によって生じる周囲流体の粘性による影響が高速化の一因となっていることが分かった.密度差を小さく,溶液濃度を濃く,落下球の球径を小さくすることで擬塑性流体中を落下する球の終端速度を遅くし,より高速化を図ることが可能であると分かった.しかし,今回の研究で用いた粘弾性を持ち,弾性による影響が無視できない流体においては,弾性による影響が支配的となると高速化が抑制されることが分かった.高速化の抑制がみられるため,高速化のピークとなる条件が存在することが示唆された.
