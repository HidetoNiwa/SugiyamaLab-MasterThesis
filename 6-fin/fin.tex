\clearpage
\section{結言}

擬塑性流体中に球を落下させるために従来,電磁石を用いて把持していた.これを真空ポンプを用いて球の把持を行うように変更した.真空ポンプを用いて球を把持した場合,初速を与えずに落下開始できることが分かった.また,従来の手法では落下開始時に初速を与えており,オーバーシュートが発生してたことが分かった.真空ポンプを用いて球を把持した場合も初速を与えて落下させるとオーバーシュートが発生することが分かった.

$De$数とオーバーシュートには正の相関があることが分かった.これにより,オーバーシュートが発生する場合,弾性による影響が生じていることが分かった.

球の落下間隔を変化させた場合,落下間隔が長くなると落下速度が遅くなった.これは落下間隔が長くなると,粘弾性の回復が十分にされるためと分かった.球の落下間隔を変化させた場合,超音波照射による高速化が見られた.高速化の要因であるが,音響境界層内の粘度とその厚さによって発生していることが分かった.今回は粘性による影響のみを議論したが,落下間隔が長くなると,落下開始時にオーバーシュートといった弾性による影響も見られた.このことより,弾性による影響も調査する必要があると考えられる.

