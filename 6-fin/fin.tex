\clearpage
\section{結言}

従来,球の把持に電磁石を用いていた.これを真空ポンプを用いて球の把持を行うように変更した.この際,終端速度は完全に一致しなかった.これは,粘度特性の違いによるものだと理論的に示唆された.

また従来,オーバーシュートが発生していたが,真空ポンプを用いて球を把持した場合は発生しなかった.$De$数とオーバーシュートには正の相関が存在した.このことから,オーバーシュートが弾性によって発生すると示唆された.オーバーシュートの発生要因であるが,従来手法では落下開始時に初速を与えているためだと示唆された.真空ポンプを用いて球の把持を行うことで,把持できる球の種類が増加し,初速による影響をなくすことができた.

球の落下間隔を変化させた場合,落下間隔が長くなると落下速度が遅くなった.これは落下間隔が長くなると,粘弾性の回復が十分にされるためと分かった.球の落下間隔を変化させた場合,超音波照射による高速化が見られた.高速化の要因であるが,音響境界層内の粘度とその厚さによって発生していることが示唆された.今回は粘性による影響のみを議論したが,落下間隔が長くなると,落下開始時にオーバーシュートといった弾性による影響も見られた.このことより,落下間隔の変化における弾性の影響も調査する必要があると考えられる.

