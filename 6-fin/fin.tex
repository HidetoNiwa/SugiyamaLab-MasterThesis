\clearpage
\section{まとめ}

今回,0.5,1wt.\%PAA溶液それぞれにおける超音波照射による球の落下の向上を観測することができた.より擬塑性が強い1wt.\%PAA溶液においてより顕著な高速化が観測された.
また,試行回数を経ることで,落下速度が速くなることも分かった.これは,擬塑性流体の履歴効果によるためであると考えられる.

一方,球を把持する方式を電磁ホルダを用いず,ピックアップツールを用いたところ,超音波照射による影響が見られない結果が得られることが多々あった.
今後,これら把持する方式による落下へ与える影響を考慮し,鋼球以外も落下させることができる環境を検討する必要がある.
