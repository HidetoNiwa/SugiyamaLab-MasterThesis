\clearpage
\section{結言}

今回,
\begin{itemize}
    \item 実験装置を真空ポンプを用いた把持方法に変更
    \item 球の落下間隔を5分,10分,20分と変化させる
\end{itemize}
ことによって以下の結言が得られた.

\begin{enumerate}
    \item 真空ポンプを用いて把持し落下させた場合,$De$数が小さくなり,オーバーシュートが見られにくくなることが分かった.
    \begin{itemize}
        \item 初速の有無によるためであると考えられる.
        \item 初速が存在すると,落下時のオーバーシュートとなるピーク速度が大きくなり,$De$が大きくなるためである.
    \end{itemize}    
    \item 落下球の落下速度は終端速度に達するまで単調増加する.これは,粘性力よりも弾性力による減速が小さいためである.
    \begin{itemize}
        \item これは,粘性よる減速が小さいためと考えられる.
        \item また,先行研究\cite{ref:8}と流体の粘弾性特性の比較を十分行う必要がある.
    \end{itemize}
    \item 落下間隔を変化させた場合,超音波照射に伴う高速化は,落下間隔10分,5分,20分の順で大きくなることが分かった.
    \begin{itemize}
        \item 落下時の粘度がその順に小さくなるからと見積もられるた.
        \item 落下間隔と粘度の変化に関して調査を行う必要がある.
    \end{itemize}
\end{enumerate}
