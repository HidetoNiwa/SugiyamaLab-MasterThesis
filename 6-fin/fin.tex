\clearpage
\section{結言}

今回,従来の落下球の把持方法を変化させ,真空ポンプを持ちいて球の把持を行った.その結果,従来の手法と同様の結果を得ることができた.さらに,落下開始時のオーバーシュートがあまり見られなかった.これは,
$De$数が従来の把持方法に比較して小さいため,弾性によって生じる落下速度のオーバーシュートがあまり見られなかったということが分かった.

また,落下球によって擬塑性流体の分子構造がせん断された後,粘弾性が回復する現象における,超音波照射による高速化の影響を調査するために落下間隔を変化させた.その結果,落下間隔が長くなると粘弾性の回復がされ落下速度は遅くなり,終端速度に達した.一方で超音波照射による効果は落下間隔が短くなるとより大きくなった.これは,粘弾性の回復が不十分であるため,粘性が小さくなり音響境界層が厚くなったためであると分かった.

