\clearpage
\section{結言}

今回,従来の落下球の把持方法を変化させ,真空ポンプを用いて球の把持を行った.その結果,電磁石を用いて球の把持をした場合と比較して,落下開始時のオーバーシュートがあまり見られなかった.一方で,真空ポンプを用いて球を把持した場合においても,球に初速を与えた場合,落下開始時にオーバーシュートが見られた.これらを踏まえて,電磁石を用いて球の把持を行った場合,落下開始時に初速を与えていることが分かった.真空ポンプを用いて球の把持を行った場合,初速が存在しないことも分かった.また,落下開始時のオーバーシュートは流体の弾性によるものだが,$De$数を用いることでオーバーシュートの有無を分類することができると分かった.

球の落下間隔を変化させた場合,落下間隔が長くなると,落下速度が遅くなった.これは落下間隔が長くなると,粘弾性の回復が十分にされるためだと分かった.また,球の落下間隔を変化させた場合も音響境界層内の粘度とその厚さによって,超音波照射による高速化が発生していることが分かった.今回は粘性による影響のみを議論したが,落下間隔が長くなると,落下開始時にオーバーシュートといった弾性による影響も見られた.このことより,弾性による影響も調査する必要があると考えられる.

