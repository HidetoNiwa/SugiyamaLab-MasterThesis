\section{結言}
本研究において,超音波照射された擬塑性流体中を落下する球
密度差,濃度,球径をパラメータとして実験によって

\begin{itemize}
    \item 溶液と落下球の密度差が小さくなると,落下球の終端速度は遅くなる.一方で超音波照射による高速化がより顕著となる.
    \item 溶液の濃度を上昇させた場合,落下球の終端速度は遅くなり,擬塑性の影響が強くなるため高速化がより顕著となる.しかし,濃度が高くなると高速化が抑制される.PAA溶液0.7-1.0wt.\%が高速化のピークとなる.
    \item PAA溶液1wt\%,落下球が鋼球の場合,直径を変化させた場合10[mm]が高速化のピークとなる.直径が10[mm]より大きい場合,球の落下による粘度と音響境界層内粘度の比に落下球の半径で規格化した音響境界層厚さを乗じた値と高速化度合いは正の相関関係となる.
\end{itemize}