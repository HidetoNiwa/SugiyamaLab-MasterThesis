\clearpage
\section{結言}
球の落下開始時に,オーバーシュートが発生した場合があった.この時,$De$数とオーバーシュートは正の相関が見られた.このことから,オーバーシュートが弾性による影響で発生していると考えられる.また,オーバーシュートの有無に関して,$De$数を用いて整理できることが分かった.

球の把持に電磁石を用いた方法では,オーバーシュートが発生していた.一方で,真空ポンプを用いて球を把持した場合は発生しなかった.オーバーシュートの発生要因であるが,電磁石を用いた方法では落下開始時に初速を与えているためだと分かった.初速を与えることで粘性抵抗では減速が足りず,弾性による影響が遅れて発生する.このため,オーバーシュートが発生ことが分かった.

球の落下間隔を変化させた場合,落下間隔が長くなると落下速度が遅くなった.これは落下間隔が長くなると,粘弾性の回復が十分にされるためと分かった.球の落下間隔を変化させた場合,超音波照射による高速化が見られた.高速化の要因であるが,音響境界層内の粘度とその厚さによって発生していることが示唆された.今回は粘性による影響のみを議論したが,落下間隔が長くなると,落下開始時にオーバーシュートといった弾性による影響も見られた.このことより,落下間隔の変化における弾性の影響も調査する必要があると考えられる.

