\section{落下球高速化の理論}

擬塑性流体中における球の落下メカニズムより考える.代表速度,代表長さとして,それぞれ,落下速度$U_T$,球直径$2a$を選ぶ.また,power-law (式(\ref{eq:power-low}))に従う流体の代表粘度を$k U_T/(2a)$とする.これらより,粒子レイノルズ数は次式で与えられる\cite{ref:1},\cite{ref:8-5}.
\begin{eqnarray}
    Re = \frac{\rho_1 \left(2a\right)^n U_T^{2-n}}{k} ,
\end{eqnarray}
と表される\cite{ref:1,ref:8-5}.今回の実験結果より,$\rho \approx$1000[kg/m${}^3$],$2a =$0.01[m],$U_T \approx$0.2[m/s],\\$k =$8.37[Pa$\cdot \text{s}^n]$,$n =$0.24であるので,$Re \approx$2.3と概算することができる.これより,粒子付近ではストークス流れに近いとみなす.流れの決定因子としては,慣性に比べて粘性が支配的であると考える.

本研究の系は,一様流れ中を半径$a$の球および,超音波の伝播が存在する系である.超音波の伝播に関して,音波の圧力変動の時間スケールは$O\left(10^{-5}\right)$sである.
一方で,球の落下現象は$O\left(10^{-2}\right)$m,$O\left(10^{0}\right)$sとなり,圧力変動の時間スケールは非常に短い.球の落下に関しては,落下時間スケールで粗視化した平均的な挙動に着目する.球周囲に存在する非圧縮性流体の単位体積の運動方程式について考える.球によって誘起される応力テンソルを$\bm{\sigma}$,流体の密度を$\rho$,速度ベクトルを$\bm{v}$,体積力を$\bm{X}$とすると,
\begin{eqnarray}
    \rho \frac{D\bm{v}}{Dt} = \bm{X} + \nabla \cdot \bm{\sigma} ,
    \label{eq:undou}
\end{eqnarray}
となる.また,非圧縮性流体を仮定しているため連続の式が成り立つ.
\begin{eqnarray}
    \nabla \cdot \bm{v} = 0 .
    \label{eq:renzoku}
\end{eqnarray}
粒子重心位置から見た移動座標系では,式(\ref{eq:undou})の左辺が,
\begin{eqnarray}
    \rho \frac{D\bm{v}}{Dt} = \frac{\partial \bm{v}}{\partial t} + \left(\bm{v} - \bm{U}_T \cdot \nabla \right) \bm{v} ,
    \label{eq:nabie}
\end{eqnarray}
と展開できる.流れは十分に発達し,定常状態であると仮定することで,式(\ref{eq:nabie})の右辺第1項の時間微分項は0となる.低レイノルズ数かつStokes近似を用いることにより,慣性項は粘性項より十分に小さいと仮定することができる.これにより,式(\ref{eq:nabie})の右辺第2項の慣性項は無視することができる.加えて,粒子周囲流体には密度差がないため,静水圧分を除いた圧力を使って応力${\bm \sigma}$と書くと,体積力${\bm X}$は0となる.これらより,式(\ref{eq:undou}), (\ref{eq:nabie})を用いると,
\begin{eqnarray}
    \nabla \cdot \sigma = 0 ,
    \label{eq:sigma-}
\end{eqnarray}
といった関係が導かれる.

ある球体領域において,ガウスの発散定理より,
\begin{eqnarray}
    \int_S{\bm{\sigma \cdot \bm{n}}}dS = \int_V{\nabla \cdot \bm{\sigma}}dV ,
    \label{eq:gaussian}
\end{eqnarray}
といった関係が成り立つ.ここで,$\bm{n}$は球領域表面に対する法線ベクトル,$S$は球の表面積,$V$は球の体積を表す.式(\ref{eq:sigma-})より,式(\ref{eq:gaussian})の右辺は0である.式(\ref{eq:sigma-}),(\ref{eq:gaussian})は任意の流体体積に関して成り立つため,球の表面($r = a$)と球外部の任意の領域($r > a$)において,以下の関係が成り立つ.
\begin{eqnarray}
    \int_{r=a}\bm{\sigma}\cdot\bm{e}_r dS=\int_r\bm{\sigma}\cdot\bm{e}_r dS .
    \label{eq:inte}
\end{eqnarray}
ここで,$\bm{n} = \bm{e}_r$とする.この式は,任意の領域において,面積力が釣り合うことを示す.$r = a$における時間平均応力を$\langle\sigma\rangle_a$,$r$における時間平均応力を$\langle\sigma\rangle_r$とそれぞれする.球の表面積$S=4\pi r^2$であるので,式(\ref{eq:inte})は,
\begin{eqnarray}
    4\pi a^2\langle\sigma\rangle_a = 4\pi r^2\langle\sigma\rangle_r ,
    \label{eq:sigma1}
\end{eqnarray}
となる.球表面において,球の体積力と表面力は釣り合うので,
\begin{eqnarray}
    4\pi a^2\langle\sigma\rangle_a = \frac{4}{3} \pi a^3 \Delta \rho g ,
    \label{eq:sigma2}
\end{eqnarray}
となる.ここで,球と流体の密度差$\Delta \rho$,重力加速度$g$である.式(\ref{eq:sigma1}),(\ref{eq:sigma2})より,
\begin{eqnarray}
    \langle\sigma\rangle_r = \frac{a^3\Delta\rho g}{3r^2} ,
\end{eqnarray}
となる.低レイノルズ数で粘性項が支配的であるため,
\begin{eqnarray}
    \langle\sigma\rangle_r \sim \mu \dot{\gamma} ,
    \label{eq:sigma3}
\end{eqnarray}
と概算することができる.Power-law modelが適用できる領域での議論を行っているため,式(\ref{eq:power-low}),(\ref{eq:sigma3})より,
\begin{eqnarray}
    \dot{\gamma} \sim \left(\frac{a^3\Delta\rho g}{3r^2 k}\right)^{\frac{1}{n}} ,
    \label{eq:gamma_abs}
\end{eqnarray}
と概算される.この系において,エネルギー散逸に関して考える.位置エネルギーと粘性によるエネルギー散逸が釣り合うため,単位時間あたりに系全体が失うエネルギーバランスより,以下の式が成立する.
\begin{eqnarray}
    \int_{r>a}\bar{\epsilon}dV = 4 \pi \int^\infty_a \bar{\epsilon}r^2 dr = \frac{4}{3}\pi a^3\Delta\rho g U_T ,
    \label{eq:eg}
\end{eqnarray}
ここで,$U_T$は球の終端速度,$\bar{\epsilon}$は時間平均された単位体積当たりのエネルギー散逸である.また,粘性散逸$\bar{\epsilon}$は,以下の様に概算される.
\begin{eqnarray}
    \bar{\epsilon} \sim \langle\sigma\rangle_r\dot{\gamma} \sim \mu \dot{\gamma}^2 \sim \frac{\langle\sigma\rangle_r^2}{\mu} ,
    \label{eq:eps}
\end{eqnarray}
式(\ref{eq:sigma3}),(\ref{eq:eg}),(\ref{eq:eps})より,終端速度は
\begin{eqnarray}
    U_T \sim \frac{a^3\Delta\rho g}{3}\int_a^\infty\frac{dr}{\mu r^2} ,
    \label{eq:UT0}
\end{eqnarray}
と見積もられる.式(\ref{eq:power-low}),(\ref{eq:gamma_abs}),(\ref{eq:UT0})より,終端速度は下記の様に書き直される.
\begin{eqnarray}
    U_T \sim \frac{a^3\Delta\rho g}{3}  \int^{\infty}_{a} \frac{dr}{\mu r^2} \sim \left(\frac{\Delta \rho g}{3k}\right)^{\frac{1}{n}}\frac{n}{2-n}a^{\frac{n+1}{n}} .
    \label{eq:UT}
\end{eqnarray}
